\documentclass[times, utf8, seminar, numeric]{fer}
\usepackage{ booktabs, tabularx, float, mdframed, graphicx, subcaption, url, listings, caption}
\newenvironment{packed_item}{
	\begin{itemize}
   }{\end{itemize}}
\newenvironment{packed_enum}{
   \begin{enumerate}
   }{\end{enumerate}}

\lstset{ 
  backgroundcolor=\color{white},   % choose the background color; you must add \usepackage{color} or \usepackage{xcolor}; should come as last argument
  basicstyle=\footnotesize,        % the size of the fonts that are used for the code
  breakatwhitespace=false,         % sets if automatic breaks should only happen at whitespace
  breaklines=true,                 % sets automatic line breaking
  captionpos=b,                    % sets the caption-position to bottom
  escapeinside={\%*}{*)},          % if you want to add LaTeX within your code
  extendedchars=true,              % lets you use non-ASCII characters; for 8-bits encodings only, does not work with UTF-8
  frame=single,	                   % adds a frame around the code
  keepspaces=true,                 % keeps spaces in text, useful for keeping indentation of code (possibly needs columns=flexible)
  keywordstyle=\color{blue},       % keyword style
  stepnumber=2,                    % the step between two line-numbers. If it's 1, each line will be numbered
  language=C++,
  basicstyle=\footnotesize,% basic font setting
  }



\begin{document}
% TODO: Navedite naslov rada.
\title{Poravnanje para sekvenci korištenjem HMM}

% TODO: Navedite vaše ime i prezime.
\author{Petra Dunja Grujić Ostojić , Tea Čutić}

% TODO: Navedite ime i prezime mentora.
\voditelj{izv. prof. dr. sc. Mirjana Domazet-Lošo}

\maketitle

\tableofcontents

\chapter{Uvod}
Jedan od najtežih, a time i najistraživanijih, problema u području bioinformatike je problem poravnanja dviju sekvenci. Poravnanje je ključan korak u uspoređivanju i analizi DNA, RNA i proteinskih sekvenci te omogućuje uvid u zajedničke evolucijske korijene ili indikatore genetskih bolesti.

\bigskip

Ovaj rad bavi se implementacijom algoritma za poravnanje koji koristi skrivene  Markovljeve modele (eng. Hidden Markov Models - HMM). Skriveni Markovljevi modeli su probabilistički modeli korišteni za predstavljanje stohastičkih procesa. Ovaj model sastoji se od više različitih stanja koja, u kontekstu modeliranja odnosa dvaju znakovnih nizova, mogu predstavljati stanja podudaranja, umetanja i brisanja. Vezama između tih stanja označene su vjerojatnosti prelaska iz jednog stanja u drugo ili ostanka u istom stanju. U ovom su radu korištena tri algoritma: Baum-Welch algoritam, Viterbijev algoritam te Needleman-Wunsch algoritam. Od navedenih algoritama, za algoritme Baum-Welch i Viterbi pisana je vlastita implementacija dok je algoritam Needleman-Wunsch preuzet iz javno dostupnog izvora te se koristi za procjenu kvalitete poravnanja kojeg dobijemo u našim rezultatima.

\bigskip

Drugo poglavlje rada bavi se detaljnijim opisom algoritama Baum-Welch i Viterbi te jednostavnim primjerom njihovog rada. Treće poglavlje bavi se analizom dobivenih rezultata te njihovom usporedbom s rezultatima dobivenima s pomoću algoritma Needleman-Wunsch.







\chapter{Opis algoritma}

\section{Baum-Welch algoritam}


\section{Viterbi algoritam}

 
\begin{lstlisting}[language=C++, caption={Ažuriranje vrijednosti Cv}, captionpos=b, label={k1}]
if (patt[i] = t[v]){
   Cv'=min({Cu/(u,v)eE}u{i-1})
}else{
   Cv' = 1 + min(Cv, min Cu/(u,v)eE)
}
\end{lstlisting}


\chapter{Analiza rezultata}

\nocite{*}
\bibliography{literatura}
\bibliographystyle{fer}


\end{document}
